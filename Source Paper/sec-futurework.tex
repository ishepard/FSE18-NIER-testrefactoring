%!TEX root = ./main.tex
\section{Discussion and Future Work}
\label{sec:futurework}
Automated testing has become an essential process for improving the quality of software systems~\cite{Spadini2018,Bertolino2007}. Automated tests can help ensure that production code is robust under many usage conditions and that code meets performance and security needs~\cite{Spadini2018,Bertolino2007}. Nevertheless, writing and maintaining high-quality test code is challenging and frequently considered of secondary importance~\cite{Spadini2018,Spadini,van2001refactoring}. In this paper, we analyzed what types of refactorings developers apply the most on test code, and how the maintainability of test code is related to the one on production code. We found that "Rename Method", "Rename Class" and "Move Method" are among the most used types of refactoring. In the second research question we aimed at understanding the relation between test code and production code maintainability, finding that most of the pairs were categorized as low risk: however, we also found that with the increasing of the production code categories, test code still remains of low risk. This indicates that even though the production code complexity is higher, most of the time the complexity of the test code is low. 

As for future works, we firstly aim at introducing a more sophisticated refactoring detection tool: indeed, we noticed that some commits include several different refactorings on the same piece of code, but only one type can be recognized by Refactoring-Miner. 

Furthermore we suggest finding a more sophisticated method for asserting software maintainability, perhaps considering several granularity levels, from system to method level. We have seen that there is already several methods proposed, each having their own flaws, we think the CK metrics are a good starting point for the assessment with possibly including the Hallstead complexity measures \cite{halstead1977elements} since they both give insight on the size and complexity of a system or piece of code.

A further area of investigation might be the relation between refactorings and maintainability. For example, does the refactoring of test code impact the maintainability of production code? And vice-versa? By analyzing this relation developers might be pushed to keep their test code in as good shape as their production code, since having higher quality test code has proven to give higher throughput and productivity \cite{athanasiou2011constructing}. 