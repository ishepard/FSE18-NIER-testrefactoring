%!TEX root = main.tex

\section{Introduction}
Automated testing is nowadays considered an essential process for 
improving the quality of software systems~\cite{Bertolino2007,Myers2004}, and it is
one of the most common techniques for detecting defects in 
software artifacts~\cite{laitenberger1998studying,van2001refactoring}.
As part of their programming activity, developers write and maintain test code 
continuously~\cite{van2001refactoring}. Zaidman~\etal~\cite{Zaidman2008} investigated the
co-evolution of test and production code, showing that they grow and are modified together.

Several different testing
practices are currently used by practitioners, such as Test Driven Development
\cite{erdogmus2010test}, Mocking~\cite{Spadini}, Extreme Programming \cite{lindstrom2004extreme} or
Acceptance Test-Driven Development \cite{aggarwal2014acceptance}, and many studies on the 
positive effects of testing practices on production code quality have
been carried on in the last decade~\cite{laitenberger1998studying,binder1996testing}. 
Some of these practices,
e.g. Extreme Programming, involve a variety of different refactoring methods.
Van Deursen~\etal~\cite{van2001refactoring} described some refactoring methods 
specifically for test code, such as \textit{Inline Resource}, \textit{Setup External Resource}
or \textit{Reduce Data}. The aim of these refactorings presented by Van Deursen was
to overcome a distinct set of bad smells than involves test code, the so called \textit{test smells}.
However, no studies have studied what type of refactorings developers apply on test
code
Our research deepens the knowledge on these test
refactoring methods by shining light on the impact of refactoring test code on
production code.

One should always take the related production code into consideration when
refactoring the test code. Despite the fact that you might make the code more
readable or maintainable, there is always a chance that the refactor might break
the tests. Of course not every refactor which is made on the test code is to
improve the quality of the project, there are also test code refactors which are
a result of changes which are made in the production code. Nonethless, each
refactor could have an impact on the test code and researching this impact will
allow us to determine the value of test code refactoring.

In order to keep the code readable and maintainable, the size of each of the
classes should be kept to a minimum \cite{baggen2012standardized}. This way it
is easier for the developer to understand what is happening in the code and when
a method or class has to be refactored the impact on the code will be less
significant. Beside the size of a class, the complexity also has an impact on
the maintainability of the test code. Dealing with these problems can increase
the quality of the test code and therefor increase the throughput and
productivity of the development of a project \cite{athanasiou2011constructing}.

Despite the fact that keeping the size and complexity of a class to a minimum
does have a positive effect on the project overall, in practice these methods
are not always complied with. When a project becomes bigger it starts to get
harder to keep it well structured. When the requirements of a project starts to
differ, code changes have to be applied, however these code changes are often
the reason why the maintainability or readability of the project starts to
decrease. This is often because the developers do not realize what kind of
impact refactoring test code can have on the maintainability of a project.

We decided to look at the way developers refactor their test code and if it does
have a positive effect on the related production code. By examining the test
code of multiple projects and evaluating the maintainability we can determine
which of these refactor methods are used and if they do improve the
maintainability of the production code. If the results of this research show
that improving your test code by refactoring bad written code, we might
stimulate developers to do this and improve the overall test code of a project.

During the evaluation of the multiple open-source projects we found out that
during the development of a project, the test code does get refactored a lot.
However, only a small fraction of these refactors was to actually improve the
maintainability of the code. In this paper we will explain how we analyzed these
repositories and which results came out of this evaluation.
