%!TEX root = main.tex

\section{Introduction}
Automated testing is nowadays considered an essential process for 
improving the quality of software systems~\cite{Bertolino2007,Myers2004}, and it is
one of the most common techniques for detecting defects in 
software artifacts~\cite{laitenberger1998studying,van2001refactoring}.
Several different testing
practices are currently used by practitioners, such as Test Driven Development
\cite{erdogmus2010test}, Mocking~\cite{Spadini}, Extreme Programming \cite{lindstrom2004extreme} or
Acceptance Test-Driven Development \cite{aggarwal2014acceptance}, and many studies on the 
positive effects of testing practices on production code quality have
been carried on in the last decade~\cite{laitenberger1998studying,binder1996testing}. 

As part of their programming activity, developers write and maintain test code 
continuously~\cite{van2001refactoring}. Zaidman~\etal~\cite{Zaidman2008} investigated the
co-evolution of test and production code, showing that they grow and are modified together.
Van Deursen~\etal~\cite{van2001refactoring} described some refactoring methods 
specifically for test code, such as \textit{Inline Resource}, \textit{Setup External Resource}
or \textit{Reduce Data}. The aim of these refactorings methods presented by Van Deursen were
to overcome a distinct set of bad smells than involves test code, the so called \textit{test smells}.
Even though previous studies showed that developers continuously maintain and refactor test code, 
no studies have been carried on to understand how developers do it, what type of refactorings they apply the most
and what is the relation between test code and production code maintainability. 
By means of quantitative research, we aim at shining light on these questions, whose answers
can help understanding current best practices in test refactorings, as well as
stimulate developers to improve the overall test code quality of their project.

During the evaluation of the multiple open-source projects we found out that
during the development of a project, the test code does get refactored a lot.
However, only a small fraction of these refactors was to actually improve the
maintainability of the code. In this paper we will explain how we analyzed these
repositories and which results came out of this evaluation.
