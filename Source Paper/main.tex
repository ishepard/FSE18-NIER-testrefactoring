\documentclass[sigconf,review,anonymous]{acmart}
\usepackage{makecell}
\usepackage{todonotes}
\usepackage{graphicx}
\usepackage{color}
\usepackage{subcaption}

\setcopyright{rightsretained}
\acmDOI{}
\acmISBN{}
\acmConference[Submitted to MSR '18]{International Conference on Mining Software Repositories}{May 2018}{Gothenburg, Sweden} 
\acmYear{2018}
\copyrightyear{2018}
\acmPrice{00.00}

\begin{document}

\title{The effect of refactoring test code on production code}
\author{Authors omitted for double-blind review}
\orcid{}
\affiliation{%
  \institution{Institute}
  \streetaddress{Address}
  \city{City} 
  \state{Country} 
  \postcode{}
}
\email{Email}

\newcommand{\ie}{\emph{i.e.},~}
\newcommand{\eg}{\emph{e.g.},~}
\newcommand{\etc}{\emph{etc.}~}
\newcommand{\etal}{\emph{et al.}~}

\begin{abstract}
Refactoring test code is not something developers see in general as an important part of the development process. Despite the fact that this might be true, if the maintainability of the test code gets fully neglected it can have a negative impact on the productivity and throughput of the project. This paper provides an empirical study of test refactors and their impact on the maintainability of the production code. With this research we want to indicate which kind of refactor methods developers apply on their test code, what kind of impact these refactor methods have on the maintainability of the related production code and at last what the correlation is between the maintainability of the test code and the production code.
\end{abstract}
\begin{CCSXML}
<ccs2012>
<concept>
<concept_id>10003120.10003130</concept_id>
<concept_desc>Human-centered computing~Collaborative and social computing</concept_desc>
<concept_significance>500</concept_significance>
</concept>
</ccs2012>
\end{CCSXML}
\ccsdesc[500]{Human-centered computing~Collaborative and social computing}
\keywords{software testing, automated testing, refactoring}
\maketitle

%add more inputs for more files
%\input{content/filename} (omit .tex)
%!TEX root = main.tex

\section{Introduction}
Automated testing is nowadays considered an essential process for 
improving the quality of software systems~\cite{Bertolino2007,Myers2004}, and it is
one of the most common techniques for detecting defects in 
software artifacts~\cite{laitenberger1998studying,van2001refactoring}.
Several different testing
practices are currently used by practitioners, such as Test Driven Development
\cite{erdogmus2010test}, Mocking~\cite{Spadini}, Extreme Programming \cite{lindstrom2004extreme} or
Acceptance Test-Driven Development \cite{aggarwal2014acceptance}, and many studies on the 
positive effects of testing practices on production code quality have
been carried on in the last decade~\cite{laitenberger1998studying,binder1996testing}. 

As part of their programming activity, developers write and maintain test code 
continuously~\cite{van2001refactoring}. Zaidman~\etal~\cite{Zaidman2008} investigated the
co-evolution of test and production code, showing that they grow and are modified together.
Van Deursen~\etal~\cite{van2001refactoring} described some refactoring methods 
specifically for test code, such as \textit{Inline Resource}, \textit{Setup External Resource}
or \textit{Reduce Data}. The aim of these refactorings methods presented by Van Deursen were
to overcome a distinct set of bad smells than involves test code, the so called \textit{test smells}.
Even though previous studies showed that developers continuously maintain and refactor test code, 
no studies have been carried on to understand how developers do it, what type of refactorings they apply the most
and what is the relation between test code and production code maintainability. 
By means of quantitative research, we aim at shining light on these questions, whose answers
can help understanding current best practices in test refactorings, as well as
stimulate developers to improve the overall test code quality of their project.

During the evaluation of the multiple open-source projects we found out that
during the development of a project, the test code does get refactored a lot.
However, only a small fraction of these refactors was to actually improve the
maintainability of the code. In this paper we will explain how we analyzed these
repositories and which results came out of this evaluation.

\input{sec-relatedwork}
%!TEX root = ./main.tex

\section{Methodology}
In this paper we aim to understand what type of refactorings developers apply the most on test code, as well as identifying the effects of those refactorings on production code. To this aim, we analyzed 3 open-source projects using a variety of available tools. In this section we elaborate on the chosen data-sources, research tools and the research questions we attempt to answer.

\subsection{Research Questions}
\label{rqs}
Below we will describe the research questions we will be answering in this paper.\\
\indent\textbf{RQ1} \textit{What type of refactorings do developers apply on test code?}\\
We want to identify what kind of refactoring methods developers apply to test code and see if we find any relation to the methods described in \cite{van2001refactoring}. We also want to look into the possible correlations to refactor methods applied to different types of system components.\\
\indent\textbf{RQ2} \textit{What is the relation between test code maintainability and production code maintainability?}\\
We aim to see whether there is a relation between test code maintainability and production code maintainability, so if an improvement or deterioration in test code maintainability also affects the production code. 

\subsection{Projects under investigation}
As subject systems for our study we consider 3 OSS projects and their 667 releases. The selection is driven by two main factors: firstly, since we have to run static analysis tools to different test refactorings and compute maintainability metrics, we focus on projects whose source code is publicly available (i.e., OSS); secondly, we analyze systems having a big corpus of test code. After filtering on these criteria, we randomly select 3 OSS projects from the list available on GitHub~\footnote{\url{https://github.com}} having different size, a big amount of releases and with a number of JUnit test cases higher than 1,000 in all the releases.

The used projects can be found in Table~\ref{table:1}. The repositories exists out of multiple projects which are either maven or gradle projects, containing a production and a test code directory. This project setup makes it more easier to find production/test pair files, which is very important for this research. All projects have a time span of at least 5 years and have over 150 releases, which means that there is a strong possibility that multiple refactors have been done in order to maintain the code.


\begin{table}[htb]
    \caption{An overview of the analysed projects}
    \label{table:1}
    \resizebox{0.8\columnwidth}{!}{%
    \begin{tabular}{lrrr} 
     \hline 
     \textbf{Project name} & \thead{\# of prod.\\files} & \thead{\# of test\\files} & \thead{\# of\\releases} \\
     \hline
     SonarQube & 3166 & 2085 & 165 \\
     Apache Hadoop & 5880 & 2498 & 288 \\ 
     ElasticSearch & 3667 & 1328 & 214 \\
     \hline
    \end{tabular}
    }
\end{table}

\subsection{Defining Maintainability}
\label{sec:maint-metric}
Since the notion of maintainability is not defined in literature, in this work we take into consideration previous studies' definitions and we create our own version of maintainability. 
On first sight one might be tempted to use the maintainability index~\footnote{\url{https://blogs.msdn.microsoft.com/zainnab/2011/05/26/code-metrics-maintainability-index/}}, however, previous studies~\cite{sjoberg2012questioning, heitlager2007practical} have demonstrated that the metric has some unreliable properties. The maintainability index can be a tool to help a developer improve his/her code, but can not be used to conclude if a project/class has a high maintainability or not. 

Research has showed \cite{sjoberg2012questioning} that the most reliable metrics of maintainability are the size of the project and the inverse cohesion. Metrics as \textbf{LOC} (Lines of Code), \textbf{NOF} (Number of Fields) and \textbf{NOM} (Number of Methods) all give some indication about the size of a class and are thereby more reliable for determining the maintainability. As in previous studies~\cite{citationneeded}, to create our version of maintainability index we also take into consideration the complexity of a class (\textbf{WMC}).

At the end, our own maintainability analysis method is based on the LOC, NOF, WMC and NOM of a Java file. Each metric is divided into 4 categories: ``Very Low'', ``Low'', ``Medium'' and ``High'', representing the risk of being poorly maintainable. As done in previous studies~\cite{alves2010deriving}, we base the boundaries of these categories on our own dataset, by taking the 70th, 80th and 90th quantile.

% \begin{table}[!ht]
%     \centering
%     \label{categories-LOC}
%     \begin{tabular}{|l|l|}
%         \hline
%         \textbf{Condition} & \textbf{Category} \\
%         \hline
%         x < 126 & Very Low \\
%         126 < x < 173 & Low \\
%         173 < x < 281 & Medium \\
%         281 < x & High \\
%         \hline
%     \end{tabular}
%     \caption{LOC maintainability categories}
% \end{table}
% \begin{table}[!ht]
%     \centering
%     \label{categories-NOF}
%     \begin{tabular}{|l|l|}
%         \hline
%         \textbf{Condition} & \textbf{Category} \\
%         \hline
%         x < 3 & Very Low \\
%         3 < x < 5 & Low \\
%         5 < x < 9 & Medium \\
%         9 < x & High \\
%         \hline
%     \end{tabular}
%     \caption{NOF maintainability categories}
% \end{table}
% \begin{table}[!ht]
%     \centering
%     \label{categories-WMC}
%     \begin{tabular}{|l|l|}
%         \hline
%         \textbf{Condition} & \textbf{Category} \\
%         \hline
%         x < 17 & Very Low \\
%         17 < x < 26 & Low \\
%         26 < x < 46 & Medium \\
%         46 < x & High \\
%         \hline
%     \end{tabular}
%     \caption{WMC maintainability categories}
% \end{table}
% \begin{table}[!ht]
%     \centering
%     \label{categories-NOM}
%     \begin{tabular}{|l|l|}
%         \hline
%         \textbf{Condition} & \textbf{Category} \\
%         \hline
%         x < 9 & Very Low \\
%         9 < x < 13 & Low \\
%         13 < x < 20 & Medium \\
%         20 < x & High \\
%         \hline
%     \end{tabular}
%     \caption{NOM maintainability categories}
% \end{table}

\subsection{Data Extraction}
\label{data-extraction}
To answer our research questions, we extracted information regarding the maintainability of a project and the types of refactorings developers apply the most on test code. To this aim, we use 3 OSS tools (they all can be found on GitHub): 
\begin{itemize}
    \item \textbf{Repodriller}: Java framework that allows the extraction of information such as commits, modifications, diffs, and source code. We used it to mine the change history information of the subject systems
    \item \textbf{CK}: Java tool to extract code metrics of Java systems
    \item \textbf{Refactoring-miner}: library written in Java that can detect refactorings applied in the history of a Java project.
\end{itemize}

To answer RQ$_1$, namely what type of refactorings developers apply the most on test code, we run the tool Refactoring-miner on all the commits of the subject systems on the last 5 years. We decided to narrow the total amount of considered commits since this process is highly time consuming. 
% To answer RQ$_2$, we collect information regarding the impact of test code refactoring on production code. For every commit of the last 5 years, we first check if there are actually any refactors made on test code: If this is not the case we skip the whole commit because it does not contain relevant information for this research question. Instead, if the commit does contain refactors, for every refactored test file we fetch the matching production class and calculate the metrics of this file every 10 commits for 5 versions (\ie we calculate the metrics of the file after 10, 20, 30, 40, 50 commits) after the test refactoring had taken place. To match the production class we exploit a traceability technique based on naming convention, \ie, it identifies the methods under test by removing the string ‘Test’ from the method name of the JUnit test method. This technique has been previously evaluated by Sneed~\cite{sneed2004reverse}, demonstrating the highest performance (both in terms of accuracy and scalability) with respect to other traceability approaches (e.g., slicing-based approaches~\cite{qusef2014recovering}). 
To answer RQ$_2$ and study the relation between test and production code maintainability, we collect monthly metrics of the systems and then map every production file to its test file. To correctly match the two types of files, we exploit a traceability technique based on naming convention, \ie, it identifies the methods under test by removing the string ‘Test’ from the method name of the JUnit test method. This technique has been previously evaluated by Sneed~\cite{sneed2004reverse}, demonstrating the highest performance (both in terms of accuracy and scalability) with respect to other traceability approaches (e.g., slicing-based approaches~\cite{qusef2014recovering}). 
For this research question, we decided to consider all the history of the projects (instead of only 5 years as previously done for RQ$_1$), but on a monthly base. More specifically, for all the subject systems, we monthly checkout the code base and collect the metrics of the entire system using \emph{CK}. 

The details about the data extraction can be found in Table.~\ref{table:6}.

\begin{table}[htb]
\caption{Statistics of the 3 projects.}
\label{table:6}
\resizebox{\columnwidth}{!}{%
\begin{tabular}{lrrr}
 & Sonarqube & Hadoop & Elasticsearch \\
\hline
\begin{tabular}[c]{@{}l@{}}Amount of monthly\\metrics commits\end{tabular} & 55 & 77 & 45 \\ 
\hline
\begin{tabular}[c]{@{}l@{}}Average amount of\\evaluated Java files\end{tabular} & 3170 & 1466 & 3875 \\ 
\hline
\begin{tabular}[c]{@{}l@{}}Total amount of\\test refactors\end{tabular} & 6832 & 5452 & 10933\\
\hline
\end{tabular}
}
\end{table}

\input{sec-implementation}
%!TEX root = ./main.tex
\section{Results}
This section describes the results to our research questions aimed at understanding the types of refactoring developers apply the most on test code as well as their impact on production code maintainability. 

\noindent
\subsection*{RQ1: What type of refactorings do developers apply on test code?}
To find what types of refactorings developers apply the most on test code, we used Refactor-miner on our subject systems. The results can be found in Table~\ref{table:9}.
% \ifx true false

% \begin{table}[!ht]
% \centering
% \begin{tabular}{|l|l|l|l|}
% \hline
% \multicolumn{4}{|c|}{Sonarqube} \\ \hline
% Extract Method & 315 & Pull Up Attribute & 33 \\ \hline
% Move Class & 1118 & Extract Superclass & 12 \\ \hline
% Move Attribute & 341 & Push Down Method & 37 \\ \hline
% Rename Package & 6 &  Push Down Attribute & 3 \\ \hline
% Move Method & 553 & Extract Interface & 1 \\ \hline
% Inline Method & 53 & Rename Class & 820 \\ \hline
% Pull Up Method & 59 & Rename Method & 3481 \\ \hline
% \end{tabular}
% \caption{Amount of refactorings by type in sonarqube}
% \label{table:9}
% \end{table}

% \begin{table}[!ht]
% \centering
% \begin{tabular}{|l|l|l|l|}
% \hline
% \multicolumn{4}{|c|}{Hadoop} \\ \hline
% Extract Method & 1690 & Pull Up Attribute & 152 \\ \hline
% Move Class & 261 & Extract Superclass & 46 \\ \hline
% Move Attribute & 395 & Push Down Method & 2 \\ \hline
% Rename Package & 0 &  Push Down Attribute & 0 \\ \hline
% Move Method & 721 & Extract Interface & 3 \\ \hline
% Inline Method & 89 & Rename Class & 555 \\ \hline
% Pull Up Method & 175 & Rename Method & 1363 \\ \hline
% \end{tabular}
% \caption{Amount of refactorings by type in hadoop}
% \label{table:10}
% \end{table}

% \begin{table}[!ht]
% \centering
% \begin{tabular}{|l|l|l|l|}
% \hline
% \multicolumn{4}{|c|}{Elasticsearch} \\ \hline
% Extract Method & 565 & Pull Up Attribute & 23 \\ \hline
% Move Class & 1363 & Extract Superclass & 73 \\ \hline
% Move Attribute & 282 & Push Down Method & 298 \\ \hline
% Rename Package & 6 &  Push Down Attribute & 100 \\ \hline
% Move Method & 1071 & Extract Interface & 0 \\ \hline
% Inline Method & 57 & Rename Class & 2401 \\ \hline
% Pull Up Method & 311 & Rename Method & 4370 \\ \hline
% \end{tabular}
% \caption{Amount of refactorings by type in elasticsearch}
% \label{table:11}
% \end{table}

% \fi

\begin{table}[!ht]
\caption{Refactorings by type in the subject systems.}
\label{table:9}
\resizebox{\columnwidth}{!}{%
\begin{tabular}{lrrrrrr}
\hline
 & Sonarqube & Hadoop & Elasticsearch & Total \\ \hline
Rename Method & 3481 & 1363 & 4370 & 9214 \\
Move Class & 1118 & 261 & 1363 & 2742 \\
Rename Class & 820 & 555 & 2401 & 3776 \\
Move Method & 553 & 721 & 1071 & 2345 \\
Move Attribute & 341 & 395 & 282 & 1018 \\
Extract Method & 315 & 1690 & 565 & 2570 \\
Pull Up Attribute & 33 & 152 & 23 & 208 \\
Extract Superclass & 12 & 46 & 73 & 131 \\
Push Down Method & 37 & 2 & 298 & 337 \\
Rename Package & 6 & 0 & 6 & 12 \\
Push Down Attribute & 3 & 0 & 100 & 103 \\
Extract Interface & 1 & 3 & 0 & 4 \\
Inline Method & 53 & 89 & 57 & 199 \\
Pull Up Method & 59 & 175 & 311 & 545 \\
\end{tabular}
}
\end{table}

\begin{figure}[!ht]
 \centering
 \includegraphics[width=\columnwidth]{resources/refactorMethods.pdf}
 \caption{Percentages of the refactor methods of all the repositories}
 \label{figure:piechart}
\end{figure}

As we can notice, some refactor methods were used more often than others. The most popular refactor method for test code would be \textit{Rename Method}. The refactor methods \textit{Rename Class}, \textit{Move Class} and \textit{Extract Method} were also used often. These methods are used for improving the names and location of codes, but also for giving the code a more logical structure. Refactor methods like \textit{Rename Package}, \textit{Extract Interface}, \textit{Extract Superclass}, \textit{Push Down Method} and \textit{Push Down Attribute} were instead barely used. Another result we can notice is that some of the refactor methods were used much more in a project than in the others. For instance, \textit{Extract Method} was used much more in the Hadoop than in Sonarqube or Elasticsearch. We speculate that this might happen because each project has their own "rules of code quality" that developers have to apply, as well as their own ideas about refactoring code (for example, some projects can have more strict rules and force developers to refactor test code).

Given our experimental setting, we cannot speculate on the motivations behind the results achieved so far: indeed, our RQ$_1$ meant to be a coarse-grained investigation aimed at understanding what types of refactorings developers apply the most. Thus, in this research question we did not focus on the reasons behind these refactorings, \eg are the test refactored due to a change in the production code? or maybe because developers were close to a new release? Our RQ$_3$ makes a first step in providing additional insights on such a relationship, however further research should better investigate this relationship.

%In the \textit{Sonarqube} project, the refactor method \textit{Rename Method} seems to be used far more often than in the \textit{Hadoop} project. The same can be said about the refactor method \textit{Extract Method}, which is used far more often in the Hadoop project as in the Sonarqube project. Ofcourse, each project has its own difficulties and thereby their own refactorings. 

%-----------------------------------------------------------------------------------------------
% \subsection*{RQ2: Does the refactoring of test code affect the maintainability of production code?}\label{maintainability:improved}
% In this section we want to adres the results found that have been collected in an attempt to answer RQ2. Refactor data on test classes has been gathered and their matching production classes were checked every 10 commits after that for 5 versions in total, so up to 50 commits ahead. As mentioned in the explanation of RQ2 we will be looking at refactors on test code that significantly improved the maintainability of that test class. We define a significant improvement as a summed difference in the classes each metric among LOC, NOF, NOM and WMC. An example, a class that went from Medium to Low in terms of LOC, has a difference of 1 in terms of LOC. High to Very Low is a difference of 4, Low to High is a difference of -2. Adding the difference for each metric gives us the summed difference, if this value exceeds 4 we say the improvement is significant.  Among all projects we have found the follwing numbers:
% \begin{table}[!ht]
%     \centering
%     \begin{tabular}{|l|l|}
%         \hline
%         \multicolumn{2}{|c|}{Test refactorings} \\ \hline
%         Total & 1756 \\ \hline
%         Worsened & 236 \\ \hline
%         Unaffected & 824 \\ \hline
%         Improved & 696 \\ \hline
%         Significantly improved & 108 \\ \hline
%     \end{tabular}
%     \caption{Counts of test refactorings and how they affected maintainability}
%     \label{table:12}
% \end{table}
% We then check the first and last version of the matching production class that was tracked and calculate the difference here as well. Some production files could not be tracked, this can be accounted to rename or deletions of the file causing our code to be unable to find the file.
% \begin{table}[!ht]
%     \centering
%     \begin{tabular}{|l|l|}
%         \hline
%         \multicolumn{2}{|c|}{Tracked production files} \\ \hline
%         Total & 108 \\ \hline
%         Affected & 0 \\ \hline
%         Unaffected & 85 \\ \hline
%         Error in tracking & 23 \\ \hline
%     \end{tabular}
%     \caption{Tracked production files' maintainability after significant maintainability improving test refactor}
%     \label{table:13}
% \end{table}
% Surprising not a single production file was modified significantly (i.e. causing a shift in its classification among the metrics we look at). Suggesting that it is very unlikely that a refactoring in test code that significantly improves maintainability causes a similar improvement in the production class under test. Now this result is not what we were expecting, now this is also something that might be related to the projects under analysis, updating test code might not happen before updating the production class, perhaps it is the other way around, production code first, after which test code gets updated. Perhaps it happens more in a periods of pure testing and pure development, such period have also been identified by \cite{zaidman2008mining}. This way our method by looking up to 50 commits in the future after a test code refactoring, does not find the corresponding production file change, if present at all. On the other hand if testing and development are done in different periods then we cannot accredit a test code refactoring to cause a change in production code.

%-----------------------------------------------------------------------------------------------
\subsection*{RQ2: What is the correlation between test code maintainability and production code maintainability?}
\label{maintainability:correlation}
Using the categories defined in Section~\ref{sec:maint-metric}, we have categorized the risk for each pair Production-Test class for all the metrics. As shown in Table~\ref{table:6}, we mined a total of 177 snapshots: Since plotting all the snapshots would result in too much data, we have chosen to visualise only 5 snapshots (randomly selected) for each project, hence 15 in total. 

In Figure~\ref{fig:heat_map} we show the result. On the axis we can find the risk categories: The X-axis shows the category in which the test class is classified, while the Y-axis shows the category of the corresponding production class. 

\subsubsection{Analyzing the results}
Throughout all the metrics we have analyzed in every project we see a strong correlation in production and test classes both being classified in the Very Low risk category. This correlation is not that present in the other categories, and this holds for each metric. We do however notice a high density in the leftmost column in each plot: This indicates that where production classes are being classified in a variety of categories, the vast majority of test classes is classified in the Very Low risk category. This can be attributed to the general lower complexity of a test class compared to a production class. For example, a highly complex production class can still be tested with a relatively simple test class that verifies if the output of certain functions is as expected.

The high density in the Very Low category for both test and production classes is a surprising result: for each of these metrics it holds that where a production class is categorized as low risk, so is the corresponding test class, giving an indication that a relation is present. However we do not see this density in the "Low/Low", "Medium/Medium" and "High/High" categories: As previously discussed, this can be led back to the argument presented earlier, that even higher complexity classes can be tested using relatively simple test classes.

\begin{figure*}
    \centering
    \includegraphics[width=\textwidth]{resources/heat_map.pdf}
    \caption{Categorization of test-production code class pairs for each metric. On the X-axis we find the test categories, while on Y-axis we find the categories of the corresponding production file.}
    \label{fig:heat_map}
\end{figure*}

\input{sec-conclusion}
%!TEX root = ./main.tex
\section{Discussion and Future Work}
\label{sec:futurework}
Automated testing has become an essential process for improving the quality of software systems~\cite{Spadini2018,Bertolino2007}. Automated tests can help ensure that production code is robust under many usage conditions and that code meets performance and security needs~\cite{Spadini2018,Bertolino2007}. Nevertheless, writing and maintaining high-quality test code is challenging and frequently considered of secondary importance~\cite{Spadini2018,Spadini,van2001refactoring}. In this paper, we analyzed what types of refactorings developers apply the most on test code, and how the maintainability of test code is related to the one on production code. We found that "Rename Method", "Rename Class" and "Move Method" are among the most used types of refactoring. In the second research question we aimed at understanding the relation between test code and production code maintainability, finding that most of the pairs were categorized as low risk: however, we also found that with the increasing of the production code categories, test code still remains of low risk. This indicates that even though the production code complexity is higher, most of the time the complexity of the test code is low. 

As for future works, we firstly aim at introducing a more sophisticated refactoring detection tool: indeed, we noticed that some commits include several different refactorings on the same piece of code, but only one type can be recognized by Refactoring-Miner. 

Furthermore we suggest finding a more sophisticated method for asserting software maintainability, perhaps considering several granularity levels, from system to method level. We have seen that there is already several methods proposed, each having their own flaws, we think the CK metrics are a good starting point for the assessment with possibly including the Hallstead complexity measures \cite{halstead1977elements} since they both give insight on the size and complexity of a system or piece of code.

A further area of investigation might be the relation between refactorings and maintainability. For example, does the refactoring of test code impact the maintainability of production code? And vice-versa? By analyzing this relation developers might be pushed to keep their test code in as good shape as their production code, since having higher quality test code has proven to give higher throughput and productivity \cite{athanasiou2011constructing}. 
\input{sec-acknowledgements}

\bibliographystyle{abbrv}
\bibliography{bibliography}

\end{document}